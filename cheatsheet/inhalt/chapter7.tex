\section{Matrix examples}
\subsection{Geometric transformations}
\textbullet \textit{Scaling}: $y = Ax$ with $A = aI$ stretches a vector by the factor $|a|$ (or shrinks it when $|a| < 1$), and it flips the vector (reverses its direction) if $a < 0$\\
\textbullet \textit{Dilation}: $y = Dx$, where D is a diagonal matrix, $D = diag(d1,d2)$. Stretches the vector x by different factors along the two different axes. (Or shrinks, if $|d_i| < 1$, and flips, if $di < 0$.)\\
\textbullet \textit{Rotation Matrix}: 
$
y = 
\begin{bmatrix}
    cos \theta & -sin \theta \\
    sin \theta & cos \theta
\end{bmatrix}x
$\\
\textbullet \textit{Reflection}
Suppose that y is the vector obtained by reflecting x through the line that passes through the origin, inclined $\theta$ radians with respect to horizontal.\\
$
y = 
\begin{bmatrix}
    cos (2\theta) & sin(2\theta) \\
    sin (2\theta) & -cos (2\theta)
\end{bmatrix}x
$\\
\textbullet \textit{Projection into a line}
Projection of point x onto a set is the point in the set that is closest to x.\\
$
y = 
\begin{bmatrix}
    (1/2)(1 + cos (2\theta)) & (1/2)sin(2\theta) \\
    (1/2)sin (2\theta) & (1/2)(1 - cos (2\theta))
\end{bmatrix}x
$
\subsection{Selectors}
An $m \times n$ selector matrix A is one in which each row is a unit vector (transposed):
\[
\begin{bmatrix}
    e^T_{k_1}\\
    .\\
    .\\
    e^T_{k_m}\\
\end{bmatrix}
\]
When it multiplies a vector, it simply copies the $k_i$th entry of x into the $i$th entry of $y = Ax$:\\
$y = (x_{k_1},x_{k_2},...,x_{k_m})$\\

\subsection{Incidence matrix}
\textbf{Directed graph}: A \textit{directed graph} consists of a set of \textit{vertices} (or nodes), labeled 1,...,n, and a set of \textit{directed edges} (or branches), labeled 1,...,m. \\
$
A_{ij} = \left\{
  \begin{array}{@{}ll@{}}
    1, & \text{edge j points to node i} \\
    -1, & \text{edge j points from node i}\\
    0, & \text{otherwise}
  \end{array}\right.
$\\
\subsection{Convolution}
